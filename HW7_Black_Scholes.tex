\documentclass{article}
\title{Black-Scholes Formula}
\author{Brian Manion}
\date{May 17, 2013}
\usepackage{mathtools}
\begin{document}
\maketitle
\section{Introduction}
    Black-Scholes or sometimes Black-Scholes-Merton is a mathematical model that seeks to explain the behavior of financial derivatives, most commonly options. 
    It was proposed by Black, Scholes, and Merton in 1972. It gave theoretical support for trading options to hedge positions, 
	which had been practice but lacked solid support. From the model we are able to calculate what the price of an option should be based on a number of 
	different factors. Nowadays there are numerous variations of the Black-Scholes model, each of which seeks to improve the model based on certain criteria, 
	usually at the cost of a significant increase in complexity. This paper will focus on the original model, the basis for all other models.
\section{Notation}
There is a bit of notation, we're going to lay it down before we get to the equations.
\begin{itemize}
\item[] C = Call option price
\item[] S = Current stock price (spot price)
\item[] K = Strike price of the option
\item[] r = risk-free interest rate
\item[] $\sigma$ = volatility of the stocks return
\item[] t = time to option maturity
\item[] N = normal cumulative distribution function
\end{itemize}

\section{Black-Scholes Equation}
The Black-Scholes equation describes the price of an option over time. The derivation of this equation is complex and exceeds the scope of this paper 
and so we simply provide the equation.


\begin{equation}
\frac{\partial \mathrm C}{ \partial \mathrm t } + \frac{1}{2}\sigma^{2} \mathrm S^{2} \frac{\partial^{2} \mathrm C}{\partial \mathrm C^2}
+ \mathrm r \mathrm S \frac{\partial \mathrm C}{\partial \mathrm S}\ =
\mathrm r \mathrm C 
\label{eq:1}
\end{equation}

Notice that that ~\eqref{eq:1} is a partial differential equation. The solution gives us the formula for the price of a put option.

\section{Black-Scholes formula for European option price}
\begin{equation}
\mathrm C(\mathrm S,\mathrm t)= \mathrm N(\mathrm d_1)\mathrm S - \mathrm N(\mathrm d_2) \mathrm K \mathrm e^{-rt}
\end{equation}

\begin{equation}
\mathrm d_1= \frac{1}{\sigma \sqrt{\mathrm t}} \left[\ln{\left(\frac{S}{K}}\right) + t\left(r + \frac{\sigma^2}{2} \right) \right]
\end{equation}

\begin{equation}
\mathrm d_2= \frac{1}{\sigma \sqrt{\mathrm t}} \left[\ln{\left(\frac{S}{K}}\right) + t\left(r - \frac{\sigma^2}{2} \right) \right]
\end{equation}

\begin{equation}
N(x)=\frac{1}{\sqrt{2\pi}} \int_{-\infty}^{x} \mathrm e^{-\frac{1}{2}z^2} dz
\end{equation}

\section{Example}
You want buy an IBM European call option with a strike price of \$210. The current stock price is \$208.99 . You calculate the volatility of the stock to be 17\%. 
The rate at which you can borrow and lend money is 5\% (this is the risk-free interest rate). 
The time to maturity of the option is 77 days. What price should you pay (per share) for the option contract?

$$\mathrm d_1= \frac{1}{0.17 \sqrt{0.21095}} \left[\ln{\left(\frac{208.99}{210}}\right) + 0.21095\left(0.05 + \frac{0.17^2}{2} \right) \right]=0.1123799$$
$$\mathrm d_2= \frac{1}{0.17 \sqrt{0.21095}} \left[\ln{\left(\frac{208.99}{210}}\right) + 0.21095\left(0.05 - \frac{0.17^2}{2} \right) \right]=0.0343001$$

Okay now that we've calculated $\mathrm d_1$ and $\mathrm d_2$ we will next calculate $\mathrm N(\mathrm d_1)$ and $\mathrm N(\mathrm d_2)$. However it should be
Okay now that we've calculated $\mathrm d_1$ and $\mathrm d_2$ we will next calculate $\mathrm N(\mathrm d_1)$ and $\mathrm N(\mathrm d_2)$. It should be noted that it's not possible to evaluate the following integrals by normal means so it is easiest to either use a table or use computer software to evaluate the
following:

$$N(d_1)=N(0.1123799)=\frac{1}{\sqrt{2\pi}} \int_{-\infty}^{0.1123799} \mathrm e^{-\frac{1}{2}z^2} dz = 0.3516077$$
$$N(d_2)=N(0.0343001)=\frac{1}{\sqrt{2\pi}} \int_{-\infty}^{0.0343001} \mathrm e^{-\frac{1}{2}z^2} dz = 0.3135993$$

Now we plug these in REFERENCE to calculate the price of the call option (per share)

$$\mathrm C(208.99,0.21095)= (0.3516077)(208.99) - (0.3135993)(210) \mathrm e^{-(0.05)(0.21095)} \approx 2.3454$$

So you should pay around \$2.35 (per share) for the options contract. As a side note, options contracts are sold in lots of 100 shares, 
so this particular contract would sell for roughly \$235.



\end{document}}